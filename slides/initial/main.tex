\documentclass{beamer}

\usepackage{listings}

\newtheorem{purpose}{Purpose}

\usetheme{SEChair}

\newcommand{\skipp}[0]{\vskip 0.5cm}
\newcommand{\skippause}[0]{\skipp \pause}

\title{Benchmarking Concurrent Programs}

\author{Scott West}
\institute{ETH Z\"{u}rich}

\begin{document}
\maketitle

\begin{frame}
  \frametitle{Measure, then cut}
  One result from the parallel language study:
  SCOOP has poor performance on shared-memory benchmarks.

  \skippause

  A successful application of benchmarking: we found a problem!
\end{frame}

\begin{frame}
  \frametitle{Concurrency v. Parallelism}
  \begin{definition}[Parallelism]
    A program doing more than one thing at a time to compute a result faster.
    The result is deterministic.
  \end{definition}

  \begin{definition}[Concurrency]
    A program with multiple threads of control that which 
    interleave in a non-deterministic way.
  \end{definition}

  They are not exclusive: concurrency can be used to implement parallelism.
\end{frame}

\begin{frame}
  \frametitle {Our task}
  A performance benchmark should be a way to measure a relevant aspect
  of execution properties.
  \skippause
  Parallelism benchmarks define a task that should have the same result
  every time, and the result must arrive quickly!
  \skippause
  We aim to define a way to benchmark concurrent programs.
\end{frame}

\begin{frame}
  \frametitle{What is a concurrent benchmark?}
  Since a concurrent program executes nondeterministically,
  benchmarking will concern profiling the performance characteristics
  of various \emph{synchronization}, \emph{communication}, and \emph{coordination} 
  tasks.
\end{frame}

\begin{frame}
  \frametitle{Benchmark Overview}
  We wish to examine \emph{scalability}: how does a benchmark react to 
  increased contention/threads/problem size.

  We will examine concurrent programs with multiple patterns:

  \begin{description}
  \item[no share] threads do not share memory or coordinate.
  \item[1-N comm.] a single master thread dispatches work to multiple workers.
  \item[M-N comm.] multiple master threads dispatch work
  \item[mutex] multiple threads contend for a critical section with high contention.
  \item[benign share] threads share memory, but do not have guarded access to it.
  \end{description}
\end{frame}

\begin{frame}[fragile]
  \frametitle{Noninterference}
  \begin{purpose}
    To test the cost of running multiple threads by the runtime system.
  \end{purpose}

  Some concurrency abstractions have non-trivial departures from
  POSIX-like threads (such as green threads) and
  this such a test would determine the effect of any such abstractions.
  \skipp
  This would take the shape of running some compute heavy task
  in each thread. I,e.

  \begin{lstlisting}
    fib (30)
  \end{lstlisting}

\end{frame}

\begin{frame}
  \frametitle{1-N communication}
  \begin{purpose}
    To examine the cost of having multiple threads contending for a single event.
  \end{purpose}
  
  This test replicates a common design in server applications where a main thread
  dispatches work to multiple reader threads.
  \skipp
  The test will have a single channel of communication where the master
  dispatches integers to the workers.
  The workers continually receive and discard the integers.
\end{frame}

\begin{frame}
  \frametitle{M-N communication}
  \begin{purpose}
    To test the effect of multiple producers accessing a single consumption channel.
  \end{purpose}

  Similar to the previous test, but with multiple masters and workers at either end
  of the work distribution.
\end{frame}

\begin{frame}
  \frametitle{Mutex}
  \begin{purpose}
    To quantify the cost of gaining exclusive access to a critical section.
  \end{purpose}

  The critical section should be heavily contended as this is the case where
  performance issues tend to arise.

  The mutex operations may be noops (if a CS is not explicitly required),
  but \emph{must} update some shared state in a simple way (ie, adding 1).
\end{frame}

\begin{frame}
  \frametitle{Interfering}
  \begin{purpose}
    To examine the overhead of continually accessing a shared resource without
    any extra protection not provided by the implementation.
  \end{purpose}

  Very often there are benign dataraces, this test quantifies the performance
  cost of modifying shared memory without an explicit mutex protecting it.
\end{frame}

\begin{frame}
  \frametitle{Preliminary results}

  \begin{center}
    \begin{tabular}[h]{l|l|l}
      Benchmark    & STM (Haskell) & SCOOP \\ \hline
      No share     &     --        &--     \\
      1-N comm.    &       --      &  --   \\
      M-N comm.    & 0.478 s       & 63.747 s \\
      Mutex        & 0.067 s       & 24.235 s \\
      Benign share &        --     &   --  \\
    \end{tabular}
  \end{center}
\end{frame}

\begin{frame}
  \frametitle{Going further}
  \begin{itemize}
  \item Add more microbenchmarks, gathered from real situations.
  \item Implement the benchmarks in various concurrency approaches,
    with a focus on languages that focus on concurrency rather than
    data parallelism.
    \begin{itemize}
      \item Candidates for other approaches: Erlang, Go, Ada, Clojure, F\#, D, Java, Scala, Orc...
    \end{itemize}
  \item Validation step: use them to predict performance of larger programs.
  \end{itemize}
\end{frame}

\begin{frame}
  \frametitle{Publication}
  Rough plan: ISSTA submission early next year.
\end{frame}

\end{document}
